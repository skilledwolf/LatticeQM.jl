%% Generated by Sphinx.
\def\sphinxdocclass{report}
\documentclass[letterpaper,10pt,english]{sphinxmanual}
\ifdefined\pdfpxdimen
   \let\sphinxpxdimen\pdfpxdimen\else\newdimen\sphinxpxdimen
\fi \sphinxpxdimen=.75bp\relax

\PassOptionsToPackage{warn}{textcomp}
\usepackage[utf8]{inputenc}
\ifdefined\DeclareUnicodeCharacter
% support both utf8 and utf8x syntaxes
  \ifdefined\DeclareUnicodeCharacterAsOptional
    \def\sphinxDUC#1{\DeclareUnicodeCharacter{"#1}}
  \else
    \let\sphinxDUC\DeclareUnicodeCharacter
  \fi
  \sphinxDUC{00A0}{\nobreakspace}
  \sphinxDUC{2500}{\sphinxunichar{2500}}
  \sphinxDUC{2502}{\sphinxunichar{2502}}
  \sphinxDUC{2514}{\sphinxunichar{2514}}
  \sphinxDUC{251C}{\sphinxunichar{251C}}
  \sphinxDUC{2572}{\textbackslash}
\fi
\usepackage{cmap}
\usepackage[T1]{fontenc}
\usepackage{amsmath,amssymb,amstext}
\usepackage{babel}



\usepackage{times}
\expandafter\ifx\csname T@LGR\endcsname\relax
\else
% LGR was declared as font encoding
  \substitutefont{LGR}{\rmdefault}{cmr}
  \substitutefont{LGR}{\sfdefault}{cmss}
  \substitutefont{LGR}{\ttdefault}{cmtt}
\fi
\expandafter\ifx\csname T@X2\endcsname\relax
  \expandafter\ifx\csname T@T2A\endcsname\relax
  \else
  % T2A was declared as font encoding
    \substitutefont{T2A}{\rmdefault}{cmr}
    \substitutefont{T2A}{\sfdefault}{cmss}
    \substitutefont{T2A}{\ttdefault}{cmtt}
  \fi
\else
% X2 was declared as font encoding
  \substitutefont{X2}{\rmdefault}{cmr}
  \substitutefont{X2}{\sfdefault}{cmss}
  \substitutefont{X2}{\ttdefault}{cmtt}
\fi


\usepackage[Bjarne]{fncychap}
\usepackage{sphinx}

\fvset{fontsize=\small}
\usepackage{geometry}

% Include hyperref last.
\usepackage{hyperref}
% Fix anchor placement for figures with captions.
\usepackage{hypcap}% it must be loaded after hyperref.
% Set up styles of URL: it should be placed after hyperref.
\urlstyle{same}
\addto\captionsenglish{\renewcommand{\contentsname}{Contents:}}

\usepackage{sphinxmessages}
\setcounter{tocdepth}{1}


% Jupyter Notebook code cell colors
\definecolor{nbsphinxin}{HTML}{307FC1}
\definecolor{nbsphinxout}{HTML}{BF5B3D}
\definecolor{nbsphinx-code-bg}{HTML}{F5F5F5}
\definecolor{nbsphinx-code-border}{HTML}{E0E0E0}
\definecolor{nbsphinx-stderr}{HTML}{FFDDDD}
% ANSI colors for output streams and traceback highlighting
\definecolor{ansi-black}{HTML}{3E424D}
\definecolor{ansi-black-intense}{HTML}{282C36}
\definecolor{ansi-red}{HTML}{E75C58}
\definecolor{ansi-red-intense}{HTML}{B22B31}
\definecolor{ansi-green}{HTML}{00A250}
\definecolor{ansi-green-intense}{HTML}{007427}
\definecolor{ansi-yellow}{HTML}{DDB62B}
\definecolor{ansi-yellow-intense}{HTML}{B27D12}
\definecolor{ansi-blue}{HTML}{208FFB}
\definecolor{ansi-blue-intense}{HTML}{0065CA}
\definecolor{ansi-magenta}{HTML}{D160C4}
\definecolor{ansi-magenta-intense}{HTML}{A03196}
\definecolor{ansi-cyan}{HTML}{60C6C8}
\definecolor{ansi-cyan-intense}{HTML}{258F8F}
\definecolor{ansi-white}{HTML}{C5C1B4}
\definecolor{ansi-white-intense}{HTML}{A1A6B2}
\definecolor{ansi-default-inverse-fg}{HTML}{FFFFFF}
\definecolor{ansi-default-inverse-bg}{HTML}{000000}

% Define an environment for non-plain-text code cell outputs (e.g. images)
\makeatletter
\newenvironment{nbsphinxfancyoutput}{%
    % Avoid fatal error with framed.sty if graphics too long to fit on one page
    \let\sphinxincludegraphics\nbsphinxincludegraphics
    \nbsphinx@image@maxheight\textheight
    \advance\nbsphinx@image@maxheight -2\fboxsep   % default \fboxsep 3pt
    \advance\nbsphinx@image@maxheight -2\fboxrule  % default \fboxrule 0.4pt
    \advance\nbsphinx@image@maxheight -\baselineskip
\def\nbsphinxfcolorbox{\spx@fcolorbox{nbsphinx-code-border}{white}}%
\def\FrameCommand{\nbsphinxfcolorbox\nbsphinxfancyaddprompt\@empty}%
\def\FirstFrameCommand{\nbsphinxfcolorbox\nbsphinxfancyaddprompt\sphinxVerbatim@Continues}%
\def\MidFrameCommand{\nbsphinxfcolorbox\sphinxVerbatim@Continued\sphinxVerbatim@Continues}%
\def\LastFrameCommand{\nbsphinxfcolorbox\sphinxVerbatim@Continued\@empty}%
\MakeFramed{\advance\hsize-\width\@totalleftmargin\z@\linewidth\hsize\@setminipage}%
}{\par\unskip\@minipagefalse\endMakeFramed}
\makeatother
\newbox\nbsphinxpromptbox
\def\nbsphinxfancyaddprompt{\ifvoid\nbsphinxpromptbox\else
    \kern\fboxrule\kern\fboxsep
    \copy\nbsphinxpromptbox
    \kern-\ht\nbsphinxpromptbox\kern-\dp\nbsphinxpromptbox
    \kern-\fboxsep\kern-\fboxrule\nointerlineskip
    \fi}
\newlength\nbsphinxcodecellspacing
\setlength{\nbsphinxcodecellspacing}{0pt}

% Define support macros for attaching opening and closing lines to notebooks
\newsavebox\nbsphinxbox
\makeatletter
\newcommand{\nbsphinxstartnotebook}[1]{%
    \par
    % measure needed space
    \setbox\nbsphinxbox\vtop{{#1\par}}
    % reserve some space at bottom of page, else start new page
    \needspace{\dimexpr2.5\baselineskip+\ht\nbsphinxbox+\dp\nbsphinxbox}
    % mimick vertical spacing from \section command
      \addpenalty\@secpenalty
      \@tempskipa 3.5ex \@plus 1ex \@minus .2ex\relax
      \addvspace\@tempskipa
      {\Large\@tempskipa\baselineskip
             \advance\@tempskipa-\prevdepth
             \advance\@tempskipa-\ht\nbsphinxbox
             \ifdim\@tempskipa>\z@
               \vskip \@tempskipa
             \fi}
    \unvbox\nbsphinxbox
    % if notebook starts with a \section, prevent it from adding extra space
    \@nobreaktrue\everypar{\@nobreakfalse\everypar{}}%
    % compensate the parskip which will get inserted by next paragraph
    \nobreak\vskip-\parskip
    % do not break here
    \nobreak
}% end of \nbsphinxstartnotebook

\newcommand{\nbsphinxstopnotebook}[1]{%
    \par
    % measure needed space
    \setbox\nbsphinxbox\vbox{{#1\par}}
    \nobreak % it updates page totals
    \dimen@\pagegoal
    \advance\dimen@-\pagetotal \advance\dimen@-\pagedepth
    \advance\dimen@-\ht\nbsphinxbox \advance\dimen@-\dp\nbsphinxbox
    \ifdim\dimen@<\z@
      % little space left
      \unvbox\nbsphinxbox
      \kern-.8\baselineskip
      \nobreak\vskip\z@\@plus1fil
      \penalty100
      \vskip\z@\@plus-1fil
      \kern.8\baselineskip
    \else
      \unvbox\nbsphinxbox
    \fi
}% end of \nbsphinxstopnotebook

% Ensure height of an included graphics fits in nbsphinxfancyoutput frame
\newdimen\nbsphinx@image@maxheight % set in nbsphinxfancyoutput environment
\newcommand*{\nbsphinxincludegraphics}[2][]{%
    \gdef\spx@includegraphics@options{#1}%
    \setbox\spx@image@box\hbox{\includegraphics[#1,draft]{#2}}%
    \in@false
    \ifdim \wd\spx@image@box>\linewidth
      \g@addto@macro\spx@includegraphics@options{,width=\linewidth}%
      \in@true
    \fi
    % no rotation, no need to worry about depth
    \ifdim \ht\spx@image@box>\nbsphinx@image@maxheight
      \g@addto@macro\spx@includegraphics@options{,height=\nbsphinx@image@maxheight}%
      \in@true
    \fi
    \ifin@
      \g@addto@macro\spx@includegraphics@options{,keepaspectratio}%
    \fi
    \setbox\spx@image@box\box\voidb@x % clear memory
    \expandafter\includegraphics\expandafter[\spx@includegraphics@options]{#2}%
}% end of "\MakeFrame"-safe variant of \sphinxincludegraphics
\makeatother



\title{LatticeQM.jl}
\date{Jan 03, 2020}
\release{alpha}
\author{Tobias M.\@{} R.\@{} Wolf}
\newcommand{\sphinxlogo}{\vbox{}}
\renewcommand{\releasename}{Release}
\makeindex
\begin{document}

\pagestyle{empty}
\sphinxmaketitle
\pagestyle{plain}
\sphinxtableofcontents
\pagestyle{normal}
\phantomsection\label{\detokenize{index::doc}}



\chapter{Getting started}
\label{\detokenize{getting_started:getting-started}}\label{\detokenize{getting_started::doc}}

\section{Installation}
\label{\detokenize{getting_started:installation}}\begin{itemize}
\item {} 
Make sure \sphinxhref{https://julialang.org}{Julia} is installed and is at least version 1.2.

\item {} 
Use Julia’s builtin package manager \sphinxtitleref{Pkg} to add the git repository
\sphinxhref{https://gitlab.ethz.ch/wolft/LatticeQM.jl}{LatticeQM.jl}. Note that at the
time of writing, the repository is private and requires login credentials.
Just contact me if you want access.

\sphinxstylestrong{Option 1:} Start an interactive julia session (\sphinxtitleref{REPL}), hit the \sphinxcode{\sphinxupquote{{]}}} key
and run:

\begin{sphinxVerbatim}[commandchars=\\\{\}]
\PYG{n}{add} \PYG{n}{https}\PYG{p}{:}\PYG{o}{/}\PYG{o}{/}\PYG{n}{gitlab}\PYG{o}{.}\PYG{n}{ethz}\PYG{o}{.}\PYG{n}{ch}\PYG{o}{/}\PYG{n}{wolft}\PYG{o}{/}\PYG{n}{LatticeQM}\PYG{o}{.}\PYG{n}{jl}
\end{sphinxVerbatim}

Once the installation is done, leave \sphinxtitleref{Pkg mode} with the backspace key.

\sphinxstylestrong{Option 2:} Execute the julia code

\begin{sphinxVerbatim}[commandchars=\\\{\}]
\PYG{k}{using} \PYG{n}{Pkg}
\PYG{n}{Pkg}\PYG{o}{.}\PYG{n}{add}\PYG{p}{(}\PYG{l+s}{\PYGZdq{}}\PYG{l+s}{h}\PYG{l+s}{t}\PYG{l+s}{t}\PYG{l+s}{p}\PYG{l+s}{s}\PYG{l+s}{:}\PYG{l+s}{/}\PYG{l+s}{/}\PYG{l+s}{g}\PYG{l+s}{i}\PYG{l+s}{t}\PYG{l+s}{l}\PYG{l+s}{a}\PYG{l+s}{b}\PYG{l+s}{.}\PYG{l+s}{e}\PYG{l+s}{t}\PYG{l+s}{h}\PYG{l+s}{z}\PYG{l+s}{.}\PYG{l+s}{c}\PYG{l+s}{h}\PYG{l+s}{/}\PYG{l+s}{w}\PYG{l+s}{o}\PYG{l+s}{l}\PYG{l+s}{f}\PYG{l+s}{t}\PYG{l+s}{/}\PYG{l+s}{L}\PYG{l+s}{a}\PYG{l+s}{t}\PYG{l+s}{t}\PYG{l+s}{i}\PYG{l+s}{c}\PYG{l+s}{e}\PYG{l+s}{Q}\PYG{l+s}{M}\PYG{l+s}{.}\PYG{l+s}{j}\PYG{l+s}{l}\PYG{l+s}{.}\PYG{l+s}{g}\PYG{l+s}{i}\PYG{l+s}{t}\PYG{l+s}{\PYGZdq{}}\PYG{p}{)}
\end{sphinxVerbatim}

\end{itemize}


\section{Usage}
\label{\detokenize{getting_started:usage}}
\begin{sphinxVerbatim}[commandchars=\\\{\}]
\PYG{k}{using} \PYG{n}{LatticeQM}
\end{sphinxVerbatim}

Note that the first import may take quite long (due to compilation). If you get
error messages about third-party packages try to install them manually through
the Pkg package manager. If it ran successfully, then you are good to go.

First-time users  may want to check out the tutorials and examples.


\section{Example code}
\label{\detokenize{getting_started:example-code}}
\begin{sphinxVerbatim}[commandchars=\\\{\}]
\PYG{c}{\PYGZsh{} Imports}
\PYG{k}{using} \PYG{n}{LinearAlgebra}\PYG{p}{,} \PYG{n}{Plots}
\PYG{n}{pyplot}\PYG{p}{(}\PYG{p}{)} \PYG{c}{\PYGZsh{} use the matplotlib backend for plots}
\PYG{k}{using} \PYG{n}{LatticeQM}

\PYG{c}{\PYGZsh{} Set up lattice}
\PYG{n}{lat} \PYG{o}{=} \PYG{n}{Geometries2D}\PYG{o}{.}\PYG{n}{honeycomb}\PYG{p}{(}\PYG{p}{)}

\PYG{c}{\PYGZsh{} Get nearest\PYGZhy{}neighbor hops in the honeycomb lattice}
\PYG{n}{hops} \PYG{o}{=} \PYG{n}{Materials}\PYG{o}{.}\PYG{n}{graphene}\PYG{p}{(}\PYG{n}{lat}\PYG{p}{;} \PYG{n}{mode}\PYG{o}{=}\PYG{o}{:}\PYG{n}{nospin}\PYG{p}{)} \PYG{c}{\PYGZsh{} or mode=:spinhalf for spin\PYGZhy{}1/2}

\PYG{c}{\PYGZsh{} Compile Bloch Hamiltonian}
\PYG{n}{h} \PYG{o}{=} \PYG{n}{get\PYGZus{}bloch}\PYG{p}{(}\PYG{n}{hops}\PYG{p}{)}

\PYG{c}{\PYGZsh{} Path in k\PYGZhy{}space}
\PYG{n}{ks} \PYG{o}{=} \PYG{n}{Structure}\PYG{o}{.}\PYG{n}{get\PYGZus{}kpath}\PYG{p}{(}\PYG{n}{lat}\PYG{p}{;} \PYG{n}{num\PYGZus{}points}\PYG{o}{=}\PYG{l+m+mi}{200}\PYG{p}{)}

\PYG{c}{\PYGZsh{} Get bandstructure}
\PYG{n}{bands} \PYG{o}{=} \PYG{n}{get\PYGZus{}bands}\PYG{p}{(}\PYG{n}{h}\PYG{p}{,} \PYG{n}{ks}\PYG{p}{)}

\PYG{n}{save}\PYG{p}{(}\PYG{n}{bands}\PYG{p}{,} \PYG{l+s}{\PYGZdq{}}\PYG{l+s}{b}\PYG{l+s}{a}\PYG{l+s}{n}\PYG{l+s}{d}\PYG{l+s}{s}\PYG{l+s}{\PYGZus{}}\PYG{l+s}{e}\PYG{l+s}{x}\PYG{l+s}{a}\PYG{l+s}{m}\PYG{l+s}{p}\PYG{l+s}{l}\PYG{l+s}{e}\PYG{l+s}{.}\PYG{l+s}{h}\PYG{l+s}{5}\PYG{l+s}{\PYGZdq{}}\PYG{p}{)} \PYG{c}{\PYGZsh{} save raw data}
\PYG{n}{p} \PYG{o}{=} \PYG{n}{plot}\PYG{p}{(}\PYG{n}{bands}\PYG{p}{,} \PYG{n}{size}\PYG{o}{=}\PYG{p}{(}\PYG{l+m+mi}{330}\PYG{p}{,}\PYG{l+m+mi}{240}\PYG{p}{)}\PYG{p}{)} \PYG{c}{\PYGZsh{} plot data}
\PYG{c}{\PYGZsh{} display(p) \PYGZsh{} might need to be uncommented in non\PYGZhy{}interactive mode}

\PYG{n}{savefig}\PYG{p}{(}\PYG{l+s}{\PYGZdq{}}\PYG{l+s}{b}\PYG{l+s}{a}\PYG{l+s}{n}\PYG{l+s}{d}\PYG{l+s}{s}\PYG{l+s}{\PYGZus{}}\PYG{l+s}{e}\PYG{l+s}{x}\PYG{l+s}{a}\PYG{l+s}{m}\PYG{l+s}{p}\PYG{l+s}{l}\PYG{l+s}{e}\PYG{l+s}{.}\PYG{l+s}{p}\PYG{l+s}{d}\PYG{l+s}{f}\PYG{l+s}{\PYGZdq{}}\PYG{p}{)} \PYG{c}{\PYGZsh{} save the figure}
\end{sphinxVerbatim}


\chapter{Tutorial}
\label{\detokenize{tutorial:tutorial}}\label{\detokenize{tutorial::doc}}
The typical workflow is as follows:
\begin{enumerate}
\sphinxsetlistlabels{\arabic}{enumi}{enumii}{}{.}%
\item {} 
Import the package via \sphinxcode{\sphinxupquote{using LatticeQM}}

\item {} 
Create or load a predefined lattice object

\item {} 
Obtain a tight-binding Hamiltonian from the lattice object

\item {} 
Use exact diagonalization to calculate bandstructure,
linear response coefficients and topological invariants

\end{enumerate}

In what follows, we shall elaborate each point. Afterwards, you will have
a good idea of how to use this package.


\section{Creating a lattice}
\label{\detokenize{tutorial:creating-a-lattice}}
Text here.


\section{Getting a tight-binding Hamiltonian}
\label{\detokenize{tutorial:getting-a-tight-binding-hamiltonian}}
Text here.


\section{Bandstructure calculations}
\label{\detokenize{tutorial:bandstructure-calculations}}
Text here.


\chapter{Examples}
\label{\detokenize{examples:examples}}\label{\detokenize{examples::doc}}
This where examples will go.


\chapter{Packages}
\label{\detokenize{packages:packages}}\label{\detokenize{packages::doc}}
\sphinxcode{\sphinxupquote{LatticeQM}} provides several subpackages:
\begin{itemize}
\item {} 
\sphinxcode{\sphinxupquote{Structure}} contains several types and methods to define and
manipulate lattice vectors, unit cells and paths in reciprocal space.
It is an abstract class for many problems.

\item {} 
\sphinxcode{\sphinxupquote{Geometries2D}} contains predefined two-dimensional lattice
geometries (so far mostly honeycomb multilayers). It is a collection of
tested examples.

\item {} 
\sphinxcode{\sphinxupquote{TightBinding}} contains types and methods to define real-space hopping
Hamiltonians (non-interacting), in particular via given \sphinxcode{\sphinxupquote{Lattice}} objects.
A tight-binding Hamiltonian can then be converted in to a Bloch Hamitlonian
matrix. It is an abstract class intended to fit many problems.

\item {} 
\sphinxcode{\sphinxupquote{Materials}} is a collection of physical examples and general modifiers to
get tight-binding operators, e.g. for graphene, or zeeman fields, spin-orbit
coupling etc.

\item {} 
\sphinxcode{\sphinxupquote{BlochTools}} contains methods to perform, save and display results from
exact diagonalization, such as bandstructure, optical conductivity,
topological invariants, etc.

\end{itemize}

There is another subpackage \sphinxcode{\sphinxupquote{KPM.jl}} implementing the Kernel Polynomial Method.
It is tested and working, but for now neither integrated nor properly documented.


\chapter{Indices and tables}
\label{\detokenize{index:indices-and-tables}}\begin{itemize}
\item {} 
\DUrole{xref,std,std-ref}{genindex}

\item {} 
\DUrole{xref,std,std-ref}{modindex}

\item {} 
\DUrole{xref,std,std-ref}{search}

\end{itemize}



\renewcommand{\indexname}{Index}
\printindex
\end{document}